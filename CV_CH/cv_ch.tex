%%%%%%%%%%%%%%%%%%%%%%%%%%%%%%%%%%%%%%%%%
% Medium Length Professional CV
% LaTeX Template
% Version 2.0 (8/5/13)
%
% This template has been downloaded from:
% http://www.LaTeXTemplates.com
%
% Original author:
% Trey Hunner (http://www.treyhunner.com/)
%
% Important note:
% This template requires the resume.cls file to be in the same directory as the
% .tex file. The resume.cls file provides the resume style used for structuring the
% document.
%
%%%%%%%%%%%%%%%%%%%%%%%%%%%%%%%%%%%%%%%%%

%----------------------------------------------------------------------------------------
%	PACKAGES AND OTHER DOCUMENT CONFIGURATIONS
%----------------------------------------------------------------------------------------

\documentclass{resume_ch} % Use the custom resume.cls style
\usepackage[left=0.4 in,top=0.3in,right=0.4 in,bottom=0.6in]{geometry} % Document margins
\usepackage[colorlinks,linkcolor=black, urlcolor=black]{hyperref}
\usepackage{graphicx}
\usepackage{lipsum}
\newcommand{\tab}[1]{\hspace{.2667\textwidth}\rlap{#1}} 
\newcommand{\itab}[1]{\hspace{0em}\rlap{#1}}

\name{朴智新1} % Your name
\address{计算机视觉, 机器学习, 深度学习相关方向} % Your address
%\address{123 Pleasant Lane \\ City, State 12345} % Your secondary addess (optional)
\address{(+86) 17621504831 \\ piaozhx@shanghaitech.edu.cn \\ www.github.com/piaozhx \\ www.piaozhx.com}  % Your phone number and email

\begin{document}
%----------------------------------------------------------------------------------------
%	EDUCATION SECTION
%----------------------------------------------------------------------------------------


% \vfill
% \begin{minipage}{0.3\linewidth}
%     \includegraphics[width=\linewidth]{example-image-a}
% \end{minipage}\hfil
% \begin{minipage}{0.55\linewidth}
% \lipsum[2]
% \end{minipage}

\begin{rSection}{教育经历}

    {\bf 上海科技大学} {\em 信息科学与工程学院, 计算机科学硕士} \hfill {\bf 上海\ 中国} \\
    \href{http://www.shanghaitech.edu.cn/eng/faculty/sist/people/212.html}{\textbf{导师: 高盛华教授}}, 方向:计算机视觉,深度学习 \hfill {\em 2017年9月 - 2020年6月}
    
    {\bf 东南大学} {\em 计算机科学与工程学院, 计算机科学学士} \hfill {\bf 南京\ 中国}\\
    \href{http://cse.seu.edu.cn/people/qgl/index_en.htm}{\textbf{导师: 漆桂林教授}}, 方向:数据挖掘 \hfill {\em 2013年9月 - 2017年7月}
\end{rSection}


%----------------------------------------------------------------------------------------
%	OBJECTIVE
%----------------------------------------------------------------------------------------
\begin{rSection}{专业技能} \itemsep -5pt {} 
\item \textbf{研究方向:} Trajectory Prediction, Image Sementation, Human Pose Estimation, Object Tracking
\item \textbf{编程工具:} \textbf{Python(Pytorch, Tensorflow), C++, Matlab}, 了解JS, CSS, HTML基本知识
\item \textbf{专业知识:} Tornado(后端), Bootstrap(前端), Docker(搭建深度学习集成开发平台), Git
\end{rSection}

%----------------------------------------------------------------------------------------
%	AWARDS AND HONORS
%----------------------------------------------------------------------------------------
\begin{rSection}{获奖经历}  \itemsep -5pt {} 
    \item {\bf 省级一等奖:} 中国大学生数学建模竞赛 (\textbf{CUMCM}), 江苏赛区 \hfill {\em 2015年7月}
    \item {\bf 国际二等奖(Honorable Mention):} 美国大学生数学建模竞赛 (\textbf{MCM}) \hfill {\em 2016年2月}
    \item {\bf 省级三等奖:} 江苏省大学生程序设计竞赛 \hfill {\em 2016年5月}
    % \item {\bf 校级三等奖:} 第十二届东南大学程序设计竞赛 \hfill 2016年5月
\end{rSection} 


%----------------------------------------------------------------------------------------
%	WORK EXPERIENCE
%----------------------------------------------------------------------------------------
\begin{rSection}{工作经历}
    {\bf 上海科技大学} {\em 信息科学与工程学院}  \hfill {\bf 上海\ 中国} \\
    $\cdot$ 高性能集群(HPC)运维助理 \hfill {\em 2018年5月 - 目前}

\end{rSection} 



%----------------------------------------------------------------------------------------
%	PUBLICATIONS
%----------------------------------------------------------------------------------------
\begin{rSection}{发表论文}
    \begin{pubSubsection}{Encoding Crowd Interaction with Deep Neural Network for Pedestrian Trajectory Prediction}{2018年2月}{Yanyu X$u^*$, \textbf{Zhixin Piao}$^*$, Shenghua Gao}{(*为并列一作)}
        \item Accepted by \textbf{CVPR} 2018
    \end{pubSubsection} 

    \begin{pubSubsection}{A Simple Baseline for Unsupervised Video Object Segmentation}{2018年9月}{Jia Zheng, Weixin Luo, \textbf{Zhixin Piao}, Shenghua Gao}{}
        \item Submitting to \textbf{AAAI} 2019
    \end{pubSubsection} 
    
    \begin{pubSubsection}{Entity Linking in Web Tables with Multiple Linked Knowledge Bases}{2016年11月}{Tianxing Wu, Shengjia Yan, \textbf{Zhixin Piao}, Liang Xu, Ruiming Wang, Guilin Qi}{}
        \item Accepted by \textbf{J}oint \textbf{I}nternational \textbf{S}emantic \textbf{T}echnology Conference (\textbf{JIST}) 2016
    \end{pubSubsection} 
\end{rSection} 


%----------------------------------------------------------------------------------------
%	WORK EXPERIENCE SECTION
%----------------------------------------------------------------------------------------
\begin{rSection}{研究经历}

\begin{rSubsection}{3D Human Pose/Shape Estimation}{\bf 上海\ 中国}{科研项目, 上海科技大学}{目前}{img/Pose.png}
    \item 利用DNN在基于SMPL的人体模型上估计人体参数与相机参数
    \item 基于Video的人体纹理恢复与抖动消除
    \item 基于以上两点的人体动作迁移(Human Action Transfer)
\end{rSubsection}

\begin{rSubsection}{Encoding Crowd Interaction with Deep Neural Network for Pedestrian Trajectory Prediction}{\bf 上海\ 中国}{科研项目, 上海科技大学(相关论文已发表于CVPR2018)}{2017年10月}{img/1.png}
    \item 提出CIDNN网络提取空间与时间特征从而获取多目标Attention的行人轨迹预测算法
    \item 在多个数据集(GC, Subway by CUHK等)下达到最好性能
    \item 框架精简易于复现,运行时间短(1.91 ms/f on CPU, 0.43 ms/f on GPU)
\end{rSubsection}


\begin{rSubsection}{A Simple Baseline for Unsupervised Video Object Segmentation}{\bf 上海\ 中国}{科研项目, 上海科技大学(相关论文已提交AAAI2019)}{2018年9月}{img/vos.png}
    \item 使用Stacked-ConvLSTM和Cascade module做Unsupervised Video Object Segmentation
    \item 第一项仅使用RGB信息来模拟无监督视频对象分割的时间一致性的工作, 取代了传统光流来编码时间信息
    \item 通过语义连贯的合成数据生成策略来增强训练序列,以克服数据不足的问题
    \item 通过包括ablation study在内的多项实验验证了我们方法的有效性。
\end{rSubsection}


\begin{rSubsection}{One-Shot/Zero-Shot Video Object Segmentation}{\bf 上海\ 中国}{竞赛项目, CVPR2018 Workshop}{2018年9月}{img/DAVIS.png}
    \item 仅在给出第一帧Mask(One-Shot)或完全没有先验(Zero-Shot)的基础上的视频目标分割
    \item 尝试不同BackBone(Multi- Scale FCN, Res-U-Net, DeepLab等), 结合Template matching与Temporal propagation的算法来解决One-Shot Setting
    \item 提出一种基于ConvLSTM与Res-U-Net的算法来解决Zero-Shot Setting
\end{rSubsection}


\begin{rSubsection}{Context Awared Object Tracking By Deep Reinforcement Learning}{\bf 上海\ 中国}{课程项目, 上海科技大学(CS280 Deep Learning)}{2017年12月}{img/Tracking.png}
    \item 实现几种基于不同特征(颜色直方图、HOG等)的相关滤波目标跟踪算法
    \item 提出一种通过深度强化学习(A3C)获取全局状态来动态调整目标跟踪所使用的特征的算法
\end{rSubsection}

\begin{rSubsection}{Entity Linking in Web Tables with Multiple Linked Knowledge Bases}{\bf 南京\ 中国}{科研项目, 东南大学 (相关论文已发表于 JIST2016)}{2016年6月}{img/entity_link.png}
    \item 提出一种基于随机游走算法的表格类实体链接算法
\end{rSubsection}
        
\end{rSection} 

\begin{rSection}{开发经历}

\begin{rSubsection}{Docker Monitor and Manager System}{\bf 上海\ 中国}{运维项目, 高性能集群容器化及其web端管理系统 \href{https://github.com/piaozhx/DockerMonitor}{{\color{blue}{[Github]}}}    }{2018年9月}{img/docker.jpg}
    \item docker虚拟化集群使环境隔离, 不需要传统用户模式限制用户权限来维持生产环境
    \item 使用deepo镜像作为深度学习开发环境, 内置常见所有环境, 无需为装库烦恼
    \item 开发基于tornado, mariaDB, bootstrap, mkDocs的web端管理系统
\end{rSubsection}

\begin{rSubsection}{MiniC Compiler++ (ASM \& COE) IDE}{\bf 南京\ 中国}{课程项目, 东南大学 (计算机综合课设)   \href{https://github.com/piaozhx/CompilerIDE}{{\color{blue}{[Github]}}}     }{2016年12月}{img/CompilerIDE.jpg}
    \item 基于Lex和Yacc工具将简单的MiniC程序编译成含有57条mips指令的asm(汇编)代码
    \item 所有asm代码会进一步被编译成coe(机器码)
    \item 基于PyQt4制作跨平台IDE, 将编译步骤可视化进行
    \item miniC, asm, coe码在debug模式下将显示到同一个视图下, 便于debug
\end{rSubsection}


\begin{rSubsection}{MiniC Compiler (Lex \& Yacc) }{\bf 南京\ 中国}{课程项目, 东南大学 (编译原理课设)  \href{https://github.com/seucs/compiler}{{\color{blue}{[Github]}}}    }{2016年8月}{img/minic.png}
    \item 使用了 networks 中 MutiDiGraph可重边可有回路的有向图来表示 NFA
    \item 最后生成的 NFA、DFA 用 GraphViz 直接画出图形,可以直观方便地观察生成的结果
    \item 通过python动态添加函数的功能将yacc中的语义动作动态的加入类中,从而实现不生成临时代码来执行yacc中的语义动作
    \item 语法支持epsilon的存在,比如:int main(int a),int main ()都是合法的
\end{rSubsection}


\end{rSection}

\end{document}
