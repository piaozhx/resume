%%%%%%%%%%%%%%%%%%%%%%%%%%%%%%%%%%%%%%%%%
% Medium Length Professional CV
% LaTeX Template
% Version 2.0 (8/5/13)
%
% This template has been downloaded from:
% http://www.LaTeXTemplates.com
%
% Original author:
% Trey Hunner (http://www.treyhunner.com/)
%
% Important note:
% This template requires the resume.cls file to be in the same directory as the
% .tex file. The resume.cls file provides the resume style used for structuring the
% document.
%
%%%%%%%%%%%%%%%%%%%%%%%%%%%%%%%%%%%%%%%%%

%----------------------------------------------------------------------------------------
%	PACKAGES AND OTHER DOCUMENT CONFIGURATIONS
%----------------------------------------------------------------------------------------

\documentclass{resume} % Use the custom resume.cls style
\usepackage[left=0.4 in,top=0.3in,right=0.4 in,bottom=0.6in]{geometry} % Document margins
\usepackage[colorlinks,linkcolor=black, urlcolor=black]{hyperref}
\usepackage{graphicx}
\usepackage{lipsum}
\newcommand{\tab}[1]{\hspace{.2667\textwidth}\rlap{#1}} 
\newcommand{\itab}[1]{\hspace{0em}\rlap{#1}}


\name{Zhixin Piao} % Your name
\address{Computer Vision, Machine Learning, Deep Learning} % Your address
%\address{123 Pleasant Lane \\ City, State 12345} % Your secondary addess (optional)
\address{(+86) 17621504831 \\ piaozhx@shanghaitech.edu.cn \\ www.github.com/piaozhx \\ www.piaozhx.com}

\begin{document}
%----------------------------------------------------------------------------------------
%	EDUCATION SECTION
%----------------------------------------------------------------------------------------

\begin{rSection}{Education}

{\bf ShanghaiTech University} \hfill {\bf Shanghai China}
\\ 
School of Information Science and Technology, M.S. in Computer Science \hfill {\em Sep. 2017 - Present}
\\
Advisor: Prof. Shenghua Gao, Major in  Computer Vision

{\bf SouthEast University} \hfill {\bf Nanjing China}
\\ 
School of Computer Science and Engineering, B.S. in Computer Science \hfill {\em Sep. 2013 - Jun. 2017}
\\
Advisor: Prof. Guilin Qi, Major in Data Mining
\end{rSection}


\begin{rSection}{Skill} \itemsep -5pt {} 
    \item \textbf{Research Insterent:} Trajectory Prediction, Image Sementation, Human Pose Estimation, Object Tracking
    \item \textbf{Programming:} \textbf{Python(Pytorch, Tensorflow), C++, Matlab}, 了解JS, CSS, HTML基本知识
    \item \textbf{Knowledge:} Tornado, Bootstrap, Docker, Git
\end{rSection}
    
%----------------------------------------------------------------------------------------
%	AWARDS AND HONORS
%----------------------------------------------------------------------------------------
\begin{rSection}{AWARDS AND HONORS}  \itemsep -5pt {} 
    \item The {\bf 1st Prize} in China Undergraduate Mathematical Contest in Modeling (\textbf{CUMCM}), Jiangsu Province \hfill Jul. 2015
    \item The {\bf 2nd Prize(Honorable Mention)} in Mathematical Contest in Modeling (\textbf{MCM}), America \hfill Feb. 2016
    \item The {\bf 3rd Prize} in Collegiate Programming Contest, Jiangsu Province \hfill May. 2016
\end{rSection} 
    
    
%----------------------------------------------------------------------------------------
%	WORK EXPERIENCE
%----------------------------------------------------------------------------------------
\begin{rSection}{WORK EXPERIENCE}
    {\bf ShanghaiTech University} {\em School of Information Science and Technology}  \hfill {\bf Shanghai\ China} \\
    $\cdot$ High Performance Cluster(HPC) {\bf DevOps Assistant} \hfill {\em May. 2018 - Present}

\end{rSection} 

%----------------------------------------------------------------------------------------
%	PUBLICATIONS
%----------------------------------------------------------------------------------------
\begin{rSection}{PUBLICATIONS}
    \begin{pubSubsection}{Encoding Crowd Interaction with Deep Neural Network for Pedestrian Trajectory Prediction}{Feb. 2018}{Yanyu X$u^*$, \textbf{Zhixin Piao}$^*$, Shenghua Gao}{(* means equal contribution)}
        \item Accepted by \textbf{CVPR} 2018
    \end{pubSubsection} 
    \begin{pubSubsection}{Entity Linking in Web Tables with Multiple Linked Knowledge Bases}{Nov. 2016}{Tianxing Wu, Shengjia Yan, \textbf{Zhixin Piao}, Liang Xu, Ruiming Wang, Guilin Qi}{}
        \item Accepted by \textbf{J}oint \textbf{I}nternational \textbf{S}emantic \textbf{T}echnology Conference (\textbf{JIST}) 2016
    \end{pubSubsection} 
\end{rSection} 

% %----------------------------------------------------------------------------------------
% %	WORK EXPERIENCE SECTION
% %----------------------------------------------------------------------------------------
% \begin{rSection}{RESEARCH EXPERIENCES}

% \begin{rSubsection}{DAVIS Challenge on Video Object Segmentation}{\bf Shanghai China}{Research Project, ShanghaiTech}{Jan. 2018 - Present} 
% \item Segment objects of interests in a video when given the mask of the first frame
% \item Explore more efficient algorithm
% \end{rSubsection} 

% \end{rSection} 

\begin{rSection}{RESEARCH EXPERIENCES}

\begin{rSubsection}{Human Motion Transfer}{\bf Shanghai\ China}{Research project, ShanghaiTech University}{Present}{../img/human_motion_transfer.png}
    \item Propose Human Distangle Model(HDM) to estimation human pose, shape and texture
    \item Implement human motion transfer by HDM and multiple view GAN
    \item Totally unsupervised learning and strong interpretation for texture 
\end{rSubsection}

\begin{rSubsection}{Encoding Crowd Interaction with Deep Neural Network for Pedestrian Trajectory Prediction}{\bf Shanghai\ China}{Research Project, ShanghaiTech University(Accepted by CVPR2018)}{Oct. 2017}{../img/1.png}
    \item Propose CIDNN to extractor spatial and temporal feature form  multiple object attention
    \item Best performance on multiple popular dataset (GC, Subway by CUHK etc.)
    \item Easy to re-implement and fast(1.91 ms/f on CPU, 0.43 ms/f on GPU)
\end{rSubsection}

\begin{rSubsection}{A Simple Baseline for Unsupervised Video Object Segmentation}{\bf Shanghai\ China}{Research Project, ShanghaiTech University}{Sep. 2018}{../img/vos.png}
    \item Propose Stacked-ConvLSTM and Cascade module for unsupervised Video Object Segmentation
    \item First RGB based feature(without optical flow) work on this task
    \item a new data augmentation to overcome small dataset problem 
\end{rSubsection}

\begin{rSubsection}{Context Awared Object Tracking By Deep Reinforcement Learning}{\bf Shnaghai\ China}{Course Project, ShanghaiTech University(CS280 Deep Learning)}{Dec. 2017}{../img/Tracking.png}
    \item Implement Correlation-Filter algorithm by multiple feature(HOG, Histogram)
    \item Propose a Context Awared object tracking method by Deep Reinforcement Learning(A3C)
\end{rSubsection}

\begin{rSubsection}{Entity Linking in Web Tables with Multiple Linked Knowledge Bases}{\bf Nanjing\ China}{Research Project, SouthEast University (Accepted by JIST2016)}{Jun. 2016}{../img/entity_link.png}
    \item Propose a random-walking based algorithm for Entity Linking in web tables
\end{rSubsection}
            
\end{rSection} 

\begin{rSection}{PROGRAMMING EXPERIENCES}

\begin{rSubsection}{Docker Monitor and Manager System}{\bf Shanghai\ China}{DevOps Project  \href{https://github.com/piaozhx/DockerMonitor}{{\color{blue}{[Github]}}}    }{Sep. 2018}{../img/docker.jpg}
    \item Build a deep learning developing environment (Including Tensorflow, Pytorch, Mxnet...)
    \item Build a container manager system(based on tornado, mariaDB, bootstrap, mkDocs...)
    \item Exclude it to multiple user container system
\end{rSubsection}

\begin{rSubsection}{MiniC Compiler++ (ASM \& COE) IDE}{\bf Nanjing\ China}{Course Project, SouthEast University   \href{https://github.com/piaozhx/CompilerIDE}{{\color{blue}{[Github]}}}     }{Dec. 2016}{../img/CompilerIDE.jpg}
    \item Compile miniC to asm code(include 57 mips code)
    \item Compile asm code to coe code(a.k.a machine code)
    \item Visualize Compiler by Compiler Monitor(based on PyQt4)
\end{rSubsection}


\begin{rSubsection}{MiniC Compiler (Lex \& Yacc) }{\bf Nnajing\ China}{Course Project, SouthEast University  \href{https://github.com/seucs/compiler}{{\color{blue}{[Github]}}}    }{Aug. 2016}{../img/minic.png}
    \item Visualize NFA、DFA by GraphViz.
    \item Implement lec and yacc just by Python
\end{rSubsection}
    
\end{rSection}

\end{document}
